% !TEX root = root.tex
%% FRONTMATTER
\begin{frontmatter}

% generate title
\maketitle

\begin{abstract}

This thesis reports experimental studies of the rheology of biological materials, with an emphasis on interfacial layers, including protein layers and bacterial biofilms. Results are presented on layers formed by the proteins lysozyme and Staphylococcal nuclease at the air--water interface, on biofilms formed by Pseudomonas bacteria at the oil--water interface, and on the bulk rheological properties of fibrin and cystic fibrosis mucus with an eye toward its role as an interfacial barrier in the lung. The evolution of interfacial mechanical response through time is interpreted in terms of the changing microscopic structure of the layer.

The studies employ interfacial microrheology, which uses the motion of micrometer-scale particles embedded in the interface to probe the mechanical response of the surrounding material and infer its rheology. Passive measurements, which rely on thermal forces to drive the particles, are complemented by active measurements, in which ferromagnetic nanowires were rotated using magnetic fields. Additionally, the study of fibrin and cystic fibrosis mucus employs a novel technique using custom fluorescent particles that can be selectively ``switched on'' and used to characterize rheology and particle mobility over physiologically relevant time and distance scales.

This thesis also presents a software toolkit, developed as part of the thesis work to meet the demands of this research, that has found applications by other researchers in other areas.

This work was conducted under the supervision of Professor Robert L. Leheny and Professor Daniel H. Reich.

\vspace{1cm}

\noindent Primary Reader: Professor Robert L. Leheny\\
Secondary Reader: Professor Daniel H. Reich

\end{abstract}

\begin{acknowledgment}

First, I am grateful to Bob Leheny for his mentorship. He has been encouraging, patient, generous with his insight and his talent for communicating research, and unfailingly available and attentive. I am also grateful to Dan Reich for his incisive guidance. Together, they have given me great opportunities and provided detailed attention to prepare me to succeed. Additionally, I thank my thesis committee: Prof. Kate Stebe, Prof. Mark Robbins, Prof. Mike Bevan, and Prof. Joelle Frechette. Special thanks go to Mark, as well as Prof. Alex Szalay, for helping support and develop my interest in computing.

Many generous collaborators have enriched my work and brightened my experience. Kate Stebe welcomed me to her group and inspired me with her vigorous enthusiasm. I thank Liana Vaccari for sharing her project with me and accompanying me through the demanding rite of the 24-hour measurement. Likewise, I thank everyone in the Hanes group and in particular Ben Schuster for sharing his broad knowledge and for inviting me to contribute to his exciting projects. I am indebted to Marek Cieplak for involving me in his work and exposing me to a different aspect of our field. And I am lucky to have found Tom Caswell and Nathan Keim, who have dwelled with me in the details of particle tracking. I have enjoyed great luck in working with many talented undergraduates, including Steve Cardinali, Daniel Firester, Victor Allard, and Bilyana Tzolova, whose dedicated efforts drove much of the work presented here. I am grateful to my predecessors in the Leheny and Reich labs for the tools they built and the guidance they provided---especially Myung Han Lee, Stuart Kirschner, Nate Capallo, Hongyu Guo, and Joel Rovner. Kui Chen, Tristan Sharp, Dan Richman, Nik Hartman, and Nuala McCullagh also provided insights and deeply appreciated camaraderie.

Finally, I am grateful to my father for teaching me that childlike curiosity has no age limit, to my mother for her cheer and encouragement, to my brother for his loyal friendship and inspiring courage, to Kelly for her unwavering confidence and loving support, and to all friends and family who have made these years of study and research enjoyable.

\end{acknowledgment}

% generate table of contents
\tableofcontents

% generate list of tables
%\listoftables

% generate list of figures
%\lofimagetrue
%\twocolumn
%\begingroup
%\let\onecolumn\twocolumn
\listoffigures
%\endgroup
%\onecolumn
%\lofimagefalse

\end{frontmatter}
