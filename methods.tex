% !TEX root = root.tex
\chapter{Experimental Methods}
\chaptermark{Methods}

\section{Videomicroscopy}

\emph{This section is in large part excerpted from a manuscript in preparation for Advanced Drug Delivery Reviews. The specific excerpts in question were coauthored by D.B. Allan and Benjamin S. Schuster.}

Time-lapse videos of particles in a sample are the raw data in particle tracking experiments. Collecting videos with the highest possible quality---in short, bright particles with high contrast against a uniform background---simplifies data analysis and enables reliable interpretation. Here we address three critical factors that influence video quality: the particles, the microscope, and the camera. Optical properties of the sample under investigation also strongly impact the quality of videos that can be acquired, but those factors are typically beyond a researcher's control.

A variety of particles and illumination schemes have been used for particle tracking, including micron-sized colloids imaged via brightfield illumination [Crocker and Grier], and gold nanoparticles via darkfield microscopy [Martin]. However, we focus our review on fluorescent particles and fluorescence microscopy, which are particularly useful for identifying drug and gene delivery nanoparticles in heterogeneous biological materials. The more intensely fluorescent the particle, the more reliable the particle identification and localization. Particles can be internally labeled, either by physically encapsulating the fluorescent dye in the particle core (though this strategy is discouraged if dye leaks from the particle), or by chemically conjugating the dye to a polymer from which particles are then formulated [Ensign; Nance]. Bright, photostable, internally-labeled fluorescent polystyrene nanospheres are commercially available, and they are often used as model drug and gene nanoparticles in particle tracking studies [Hanes and Braeckmans lab papers]. Particles can alternatively be externally labeled by chemically conjugating fluorescent dye to functional groups on the particle surface. Internal labeling is often preferable, as it is less likely to change the surface properties of the particle and alter the particle�s interaction with the specimens of interest, compared to external labeling. Many fluorescent dyes, with various chemical reactivities, are commercially available, though photostable dyes with high quantum yield and extinction coefficient should be selected whenever possible [Murphy]. The color of the fluorophore is another critical consideration. Biological specimens often have high autofluorescence, especially at shorter excitation wavelengths, which can interfere with particle detection and tracking. Red or dark red fluorescent dyes may be advantageous [Rust 2011; Schuster 2014], if a light source and detection scheme optimized for such dyes are available. 
Although some research groups use microscopes specially designed and optimized for particle tracking [Braeckmans chapter], custom-built instrumentation usually is not required. In fact, particle tracking videos can be collected using hardware common in biomedical research labs and core facilities [Crocker and Hoffman]. For tracking of fluorescent particles, a widefield microscope with epifluorescence illumination and appropriate fluorescence filter sets is typically used. Objectives with large numerical aperture are ideal for high resolution particle tracking, since they collect more light, and tracking precision is proportional to the square root of the number of photons emitted by a fluorescent particle [Crocker and Hoffman]. A high-powered light source, such as a mercury or metal halide arc lamp, should be used at its highest intensity setting. Then, the exposure time of the camera should be adjusted so that the image is neither under- nor over-saturated [Crocker and Hoffman]. This procedure will minimize particle tracking errors, as discussed in a later section. The microscope should be placed on a pneumatic vibration-isolation table to dampen any external vibrations, such as from footsteps and lab equipment. If the microscope stage is to be heated for the experiment, the temperature should be at equilibrium before collecting tracking videos, to prevent sample drift arising from expansion or contraction of microscope components.
A sensitive (high quantum efficiency), low-noise camera is essential for precise particle tracking. The best choice is an electron-multiplying charge coupled device (EMCCD) camera, though scientific complementary metal oxide semiconductor (sCMOS) cameras can be used as well [Small]. The required frame rate will depend on the particle dynamics and research question. We have found 15 frames per second to be sufficiently fast for studying transport of particles with diameters on the order of 100 nm in biological fluids and tissues, including mucus [Lai, Suk, Schuster], vitreous [Xu], and brain tissue [Nance]. Faster frame rates will be necessary for studying more rapid movement [Crocker and Hoffman]. If the exposure time can be set independently of the frame rate, the exposure should be no longer than necessary to achieve bright particle images. The rationale behind this guideline will be discussed in the section on errors in particle tracking. Although particle tracking videos are large files, it is important that the data be saved in a �lossless� file format that does not compress the file by discarding information [Hoffman and Crocker].
Two rules should be kept in mind when preparing specimens for particle tracking experiments. First, when possible, the chamber containing the specimen should be well sealed to minimize sample evaporation and convective flow during the experiment [Savin and Doyle; Hanes Lab papers]. Second, the particle concentration must be appropriate for visualizing and tracking individual particles. The average particle spacing must be larger than the average particle displacement from frame to frame, or else the paths of multiple particles will intermingle, making it difficult to accurately construct individual trajectories [Crocker and Grier]. If the particles are moving very rapidly, a faster frame rate can be used, but if that is not sufficient or feasible, a new specimen with lower particle concentration should be prepared.
When performing particle tracking using unfamiliar hardware or software, it is helpful to track particles in a simple fluid with known viscosity, like water, and verify that the diffusion coefficients can be measured to within a few percent of the value predicted by the Stokes-Einstein equation [Crocker and Grier]. Note that simple systems give cleaner images than most biological systems; treat them as a best-case scenario.

\section{Wire Fabrication}
\section{Sample Prepration}
\subsection{Protein Layers}
\subsection{Bacterial Biofilms}

Borrow historical refs from Jan's "Mission Impossible" talk.