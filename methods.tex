% !TEX root = root.tex
\chapter{Experimental Methods}
\chaptermark{Methods}

\section{Particle Tracking Experimental Design}

\emph{NOTE TO BOB: Most or all of this will be removed and replaced with some writing on errors in particle tracking. That section of the review for ADDR was included in Ben's thesis, so I will need to write my own take on it.}

Time-lapse videos of particles in a sample are the raw data in particle tracking experiments. Collecting videos with the highest possible quality---in short, well defined particles with high contrast against a uniform background---simplifies data analysis and enables reliable interpretation. Here we address three critical factors that influence video quality: the particles, the microscope, and the camera. Optical properties of the sample under investigation also strongly impact the quality of videos that can be acquired, but those factors are typically beyond a researcher's control.

A variety of particles and illumination schemes have been used for particle tracking, including micron-sized colloids imaged via brightfield illumination \cite{Crocker1996}, gold nanoparticles via darkfield microscopy \cite{Martin2002a}, and fluorescent particles via fluorescence microscopy\cite{Nance2012}.

Bright, photostable, internally-labeled fluorescent polystyrene nanospheres are commercially available. Particles can alternatively be externally labeled by chemically conjugating fluorescent dye to functional groups on the particle surface. Internal labeling is often preferable, as it is less likely to change the surface properties of the particle and alter the particle�s interaction with the samples of interest, compared to external labeling. Many fluorescent dyes, with various chemical reactivities, are commercially available, though photostable dyes with high quantum yield and extinction coefficient should be selected whenever possible [Murphy]. The color of the fluorophore is another critical consideration. Biological specimens often have high autofluorescence, especially at shorter excitation wavelengths, which can interfere with particle detection and tracking. Red or dark red fluorescent dyes may be advantageous [Rust 2011; Schuster 2014], if a light source and detection scheme optimized for such dyes are available. 

Although some research groups use microscopes specially designed and optimized for particle tracking [Braeckmans chapter], custom-built instrumentation usually is not required. In fact, particle tracking videos can be collected using hardware common in condensed matter physics research labs and core facilities\cite{Crocker2007}. For tracking of fluorescent particles, a widefield microscope with epifluorescence illumination and appropriate fluorescence filter sets is typically used. Objectives with large numerical aperture are ideal for high resolution particle tracking, since they collect more light, and tracking precision is proportional to the square root of the number of photons emitted by a fluorescent particle\cite{Crocker2007}. A high-powered light source, such as a mercury or metal halide arc lamp, should be used at its highest intensity setting. Then, the exposure time of the camera should be adjusted so that the image is neither under- nor over-saturated\cite{Crocker2007}. This procedure will minimize particle tracking errors, as discussed in a later section. The microscope should be placed on a pneumatic vibration-isolation table to dampen any external vibrations, such as from footsteps and lab equipment. If the microscope stage is to be heated for the experiment, the temperature should be at equilibrium before collecting tracking videos, to prevent sample drift arising from expansion or contraction of microscope components.

In fluorescent particle-tracking experiments, a sensitive (high quantum efficiency), low-noise camera is essential for precise particle tracking. The best choice is an electron-multiplying charge coupled device (EMCCD) camera, though scientific complementary metal oxide semiconductor (sCMOS) cameras can be used as well [Small]. The required frame rate will depend on the particle dynamics and research question. Some have found 15 frames per second to be sufficiently fast for studying transport of particles with diameters on the order of 100 nm in biological fluids and tissues, including mucus [Lai, Suk, Schuster], vitreous\cite{Xu2013}, and brain tissue\cite{Nance2012}. Faster frame rates will be necessary for studying more rapid movement. If the exposure time can be set independently of the frame rate, the exposure should be no longer than necessary to achieve bright particle images. The rationale behind this guideline will be discussed in the section on errors in particle tracking. Although particle tracking videos are large files, it is important that the data be saved in a lossless file format that does not compress the file by discarding information\cite{Crocker2007}.

Two rules should be kept in mind when preparing samples for particle tracking experiments. First, when possible, the chamber containing the sample should be well sealed to minimize sample evaporation and convective flow during the experiment\cite{Savin2005}. Second, the particle concentration must be appropriate for visualizing and tracking individual particles. The average particle spacing must be larger than the average particle displacement from frame to frame, or else the paths of multiple particles will intermingle, making it difficult to accurately reconstruct individual trajectories\cite{Crocker1996}. If the particles are moving very rapidly, a faster frame rate can be used, but if that is not sufficient or feasible, a new specimen with lower particle concentration should be prepared.

\section{Ferromagnetic Nickel Nanowires}

For this work, ferromagnetic nickel nanowires were fabricated for use in active microrheology experiments. The wires are 5--35 \textmu m long with a diameter of 350 nm. Because their diameters are comparable to the characteristic size of a magnetic domain in bulk nickel and their aspect ratio is high, they exhibit a single magnetic domain over their entire volume, aligned with the long axis. Their remnant magnetization is 70\% of the saturation of bulk nickel\cite{Sun1999} with a magnetic moment per unit length of $3 \times 10^{-14}\,\text{A}\cdot\text{m}^2$/\textmu m. Thus, they couple strongly to externally applied magnetic fields, and they act as strong, robust probes of soft--matter systems.

They were fabricated by electrochemical deposition inside the pores of a nanoporous template: a ceramic filter, obtained commercially and repurposed for this technique. The fabrication protocol has been described both in published literature\cite{Chien2002} and a thesis\cite{TanaseThesis}.

\section{Tracking Nanowire Orientation}

The raw data of our active microrheology experiment is a series of video microscopy images of a nanowire. The orientation of the wire must be extracted from the images. Although the orientation is plain to any human observer, an automated solution is required due to the sheer volume of data ($10^5$ images) and the necessity of a precise and consistent judgements for providing an angular trajectory through time. The analysis is deceptively difficult to automate. We have developed a family of approaches, each with certain advantages. The important metrics are precision, robustness, and performance (speed).

A standard algorithm for identifying line segments in an image is the Hough Transform. Unfortunately, this is completely ineffective on our data. Although the wires are rods with a high aspect ratio, diffraction effects cause them to appear more like fuzzy ellipses that straight lines.

The simplest effective technique is to fit an ellipse to the wire's silhouette. A threshold defines regions inside and outside the wire, and a least-squares best fit obtains the ellipse that most accurately encloses that region. The orientation of the ellipses long axis is the orientation of the wire. This approach is robust---it almost almost captures orientation within 10$^\circ$---but it is not precise enough to track the wire's rotation smoothly.

A more precise technique is to fit a Gaussian to each individual row or column of pixels in the image, where each Gaussian's center identifies the position of the wire along that line. The linear regression of the Gaussian centers gives the centerline of the wire. This technique is precise but quite slow. Worse, it is brittle, prone to completely missing the wire orientation in noisy or otherwise non-ideal conditions.

The most effective technique is to find the ``inertial axes'' of the image (where brightness = mass). For an image $\{I_i\}$ we compute its covariance matrix $u$

\begin{equation}
\left( \begin{array}{c}
\overline{x} \\
\overline{y} \end{array} \right)
= \left( \begin{array}{c}
\sum_{i} I_i x_i / \sum_i I_i \\
\sum_{i} I_i y_i / \sum_i I_i \end{array} \right)
\end{equation}

\begin{equation}
u = \left( \begin{array}{cc}
\sum_{i} I_i x_i^2 / \sum_i I_i - \overline{x}^2 & \sum_{i} I_i x_i y_i / \sum_i I_i - \overline{x}\,\overline{y} \\
\sum_{i} I_i x_i y_i / \sum_i I_i - \overline{x}\,\overline{y} & \sum_{i} I_i y_i^2 / \sum_i I_i - \overline{y}^2\end{array} \right)
\end{equation}

\noindent from which we can obtain the orientation of the wire,

\begin{equation}
\theta = \frac{1}{2}\tan^{-1}\left(\frac{2u_{11}}{u_{20} - u_{02}}\right).
\end{equation}

\noindent The technique is much faster than fitting Gaussians, and its precision is comparable. We have also found it to be more robust, though not quite as robust as the ellipse-fitting method.

All of the methods above require standard image preparation techniques. The region of interest is isolated from the surrounding image (the wire is largest connected region of bright pixels). The image is gently blurred to suppress the jagged effect of camera noise, and background pixels are clipped to black, but the feathered edge of the wire's profile is retained.