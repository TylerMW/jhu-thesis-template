% !TEX root = root.tex
\chapter{Introduction}
\chaptermark{Introduction}

\section{Overview and Significance}

\section{Background}

\subsection{Rheology}

\subsubsection{Fundamentals}

Hooke's Law describes a solid as an idealized spring. A small deformation requires an applied force. The larger the deformation, the larger the requisite force, but the force in no way depends on the \emph{rate} of the deformation: fast and slow cost the same. A viscous liquid, as described by Newton's laws, behaves in a complementary way: it resists a small strain with a force proportional to the rate of strain but independent of the magnitude of that strain.

Real materials deviate from these simple pictures in two ways. First, if the force or deformation is large, the relationship between the applied stress and the strain (or, for liquids, strain rate) becomes more complicated than simple proportionality; the response is nonlinear. Second, even if the forces and deformations are small, some materials display both solidlike and liquidlike characteristics. For example, many materials respond elastically on short time scales but flow over long time scales. This temporal behavior reflects a spectrum of relaxation times that is dictated by the material's microstructural dynamics. Materials like these are viscoelastic.\cite{Ferry1980}

The theory of linear viscoelastic materials begins with a constitutive relation, a rheological equation of state relating stress $\sigma$ to strain $\gamma$. The equation states that changes in strain are linear. The shear modulus $G$ is a memory function representing the ratio of stress to strain through a given strain history.

\begin{equation}
  \label{eqn:linear-constituive}
  \sigma(t) = \int_{-\infty}^t G(t - t')\dot{\gamma}(t')\, dt'
\end{equation}

\noindent Let us apply this principle to a simple oscillatory strain, where a material is periodically deformed with some amplitude $\gamma_0$ at some frequency $\omega$.

\begin{equation}
  \gamma(t) = \gamma_0 \sin \omega t
\end{equation}

\noindent Eq. (\ref{eqn:linear-constituive}) gives

\begin{equation}
  \begin{aligned}
    \sigma(t) &= \int_0^\infty G(t - t') \gamma_0\,\omega \cos(\omega t')\, dt'\\
    &= \gamma_0\,\omega \int_0^\infty G(\Delta t) \cos[\omega (t - \Delta t)]\, d\Delta t\\
    &= \gamma_0 \left(\left[\omega \int_0^\infty G(\Delta t) \sin \omega \Delta t\, d\Delta t \right] \sin \omega t + \left[\omega \int_0^\infty G(\Delta t) \cos \omega \Delta t\, d\Delta t \right] \cos \omega t \right)
  \end{aligned}
\end{equation}

\noindent Because the bracketed factors depend only on $\omega$, we can define new functions $G'(\omega)$ and $G''(\omega)$, giving

\begin{equation}
  \sigma(t) = \gamma_0\, (G'(\omega) \sin \omega t + G''(\omega) \cos \omega t).
\end{equation}

\noindent The storage modulus $G'(\omega)$ gives the solidlike response, in phase with the oscillatory strain $\gamma(t)$, and the loss modulus $G''(\omega)$ gives the liquidlike response, 90$^\circ$ out of phase. They comprise the real and imaginary parts of the complex shear modulus, $G^*(\omega) = G'(\omega) + iG''(\omega)$, which is a fundamental linear response function and a quantity through which contact between theory and experiment is often achieved.

$G^*(\omega)$ provides a complete description of a material's linear shear response. Finite stresses and deformations can reveal rich, nonlinear responses, including a yield stress (some minimum stress required to deform) or shear-thinning or -thickening (where the marginal deformation decreases or increases with the applied stress).
\subsubsection{Measurement Techniques}

There is a large family of well-established techniques for measuring a material's frequency-dependent shear modulus $G^*(\omega)$. The stress--strain relationship can be probed in either direction, by applying a certain stress and measuring the resultant strain or by measuring the stress necessary to enforce a certain strain. For example, a creep experiment measures the strain in response to a sudden stress, whereas a stress relaxation experiment measures the stress following a sudden strain. Both of these are transient experiments, observing the material's response as a function of time after a brief change. Oscillatory experiments provide complimentary information, applying a periodic stress or strain at some frequency $\omega$, qualitatively probing the same response as a transient experiment at time $t=1/\omega$. Oscillatory experiments are especially important for probing short-time scale or (high-frequency) behavior.

Shear rheology measurements, whether transient or oscillatory, strain-controlled or stress-controlled, can be performed in many different geometries. Important ones include simple shear flow between sheared parallel plates or, much more commonly, between rotating coaxial cylinders (Couette flow); flow inside a rectangular slit; torsion between two parallel plates; and torsion between a cone and plate.

All of these are, of course, subject to limitations. In oscillatory experiments the inertia of the rotating tool can dominate the shear response of the sample at  high frequencies. And all experiments contend with the problem of slip, the possibility that the sample might slip along a bounding surface where its velocity was assumed to be zero with respect to the surface.

\subsection{Interfacial Rheology\label{sec:interfacial-rheology}}

Fluid--fluid interfaces exhibit the same complex, non-Newtonian, and nonlinear behavior observed in bulk materials. The central experimental challenge in characterizing these interfaces is isolating the interfacial rheological response from the response of the surrounding fluids, as the flows at the interface and the bulk fluids are intimately coupled\cite{Fuller2011}. The relative contributions of surface and bulk forces on a rheological probe is captured by the Boussinesq number\cite{Derkach2009}

\begin{equation}
  \label{eqn:simple-bo}
  \text{Bo} = \frac{\eta_s}{l(\eta_1 + \eta_2)}
\end{equation}

\noindent where $\eta_s$ is the surface viscosity and $\eta_{1,2}$ are the bulk viscosities of the fluids above and below. Oftentimes these fluids are identical ($\eta_1 + \eta_2 = 2\eta_2$) or else one is air and can be neglected ($\eta_1 + \eta_2 = \eta_2$). Surface viscosity and bulk viscosity differ by a dimension of a length, and a length scale $l$ is introduced. The definition of $l$ depends on the particulars of the geometry of the technique. Heuristically, $l$ is sometimes presented as the ratio of the interfacial surface area subtended by the probe to its perimeter of contact with the interface. But this picture of $l$ does not hold, even qualitatively, for some geometries\cite{Levine2004}.

For $Bo \ll 1$, interfacial rheology dominates the force on the rheological probe. For $Bo \gg 1$, forces from the bulk fluid dominate. Thus, a well designed technique maximizes Bo by minimizing $l$. And, unless the contribution from the bulk is negligible, the technique must be have an accompanying theory for separating the interfacial and bulk contributions.


Eq. (\ref{eqn:simple-bo}) applies to a simple viscous interface, but Bo can be generalized for viscoelastic interfaces\cite{Fuller2012}:

\begin{equation}
  \text{Bo}(\omega) = \frac{G_s''(\omega) - i G_s'(\omega)}{\omega l(\eta_1 + \eta_2)}
\end{equation}


Experimental techniques tackle the experimental challenge of isolating the interfacial response in very different ways. One is a bicone or disk, centered at the interface and operated like a traditional rotating-tool rheometer. The torque on the disks is separated into the components using methods of continuum mechanics\cite{Derkach2009}, initially worked out by Boussinesq himself\cite{Boussinesq1913} and later refined\cite{Oh1978}. The tool makes a full circular patch in the interface and couples strongly to bulk flows, diminishing its sensitivity to the interfacial rheology. This is quantified by the Boussinesq number: the length scale $l$ is related to the radius of the tool, diminishing Bo for given viscosities $\eta$, $\eta_s$.

A newer technique improves sensitivity by replacing the bicone with a double wall-ring geometry\cite{Vandebril2010}. The open center of the ring diminishes coupling to bulk flows, and the important scale $l$ becomes the thin width of the ring rather than its radius.

Other techniques rely on millimeter- to micron-scale colloidal probes manipulated by applied optical or magnetic forces. The most popular of these, dubbed the Interfacial Shear Rheometer (ISR), moves a needle through oscillatory translations along its axis\cite{Brooks1999}, gliding it across the interface with a magnetic force. Others use magnetic force to rotate a needle\cite{Lee2010} or a disk [SQUIRES REF.] or in the plane of the interface. In this work, we employ a rotating needle.

%On the Superficial... http://archive.org/stream/scientificpapers03rayliala/scientificpapers03rayliala_djvu.txt

\section{Microrheology}

Microrheology relates the motion of colloidal particles to the rheology of the surrounding medium. Often, the motion in question is thermally-driven Brownian fluctuation (``passive microrheology'') but in recent years the field has come to encompass ``active microrheology'', where colloids driven by optical or magnetic forces can probe stiffer materials and larger deformations, extending into the regime of nonlinear response.

Follow Squires here...

\subsection{Fundamentals}

\begin{equation}
  m\dot{v}(t) = f_R(t) - \int_0^t\zeta(t-t')v(t')\,dt'
\end{equation}

\begin{equation}
  m\left[s\tilde{v}(s) - v(0)\right] = \tilde{f}_R(s) - \tilde{v}(s)\tilde{\zeta}(s)
\end{equation}
\begin{equation}
  \tilde{v}(s) = \frac{\tilde{f}_R(s) + mv(0)}{ms + \tilde{\zeta}(s)}
\end{equation}

\begin{equation}
  \langle v(0)\tilde{v}(s) \rangle = \frac{\langle v(0)\tilde{f_R}(s)\rangle + m \langle v^2(0)\rangle}{ms + \tilde{\zeta}(s)}
\end{equation}

$\frac{1}{2}m\langle v^2(0) = \frac{1}{2}k_B T$

\begin{equation}
  \langle v(0)\tilde{v}(s) \rangle = \frac{Nk_B T}{ms + \tilde{\zeta}(s)} = Nk_B T \tilde{\zeta}^{-1}(s) = Nk_B T \tilde{M}(s)
\end{equation}

...get a simple expression with MSD and $G*(\omega)$...

\subsection{Interfacial Microrheology}

As described in Section \ref{sec:interfacial-rheology}, experimental techniques for measuring interfacial rheology should maximize their sensitivity to the interface relative to the bulk (i.e., maximize Bo), and they should include a procedure for separating the interfacial and bulk contributions. Interfacial microrheology faces the same issue.

\subsection{Passive Microrheology}


\subsection{Active Nanowire Microrheology}


\section{Protein Layers}
\section{Bacterial Biofilms}
\section{Cystic Fibrosis}