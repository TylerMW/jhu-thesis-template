% !TEX root = root.tex
\chapter{\label{chap:photoactivation}Multiscale Diffusion Measurements in Biological Gels Using Photoactivatable Fluorescent Nanoparticles}
\chaptermark{Photoactivation}

\section{Introduction}

Drug-delivery nanoparticles often must diffuse through biological barriers to achieve therapeutic efficacy at their target site. In some instances, such as tear film at the ocular surface, the barrier is only a few micrometers thick \cite{King-Smith2000,Azartash2011}. In other cases, nanoparticles must penetrate tens of micrometers or more through viscoelastic biological gels or tissue [papers from labs of Rakesh Jain, Hanes, maybe Saltzman and Langer too]. In vitro diffusion measurements at physiologically relevant length scales are valuable for designing nanoparticles that will exhibit favorable in vivo biodistribution.
Here, we present a strategy for measuring nanoparticle diffusion in biological gels over length scales ranging from sub-micrometer to tens of micrometers using photoactivatable fluorescent probes. This work is part of a collaboration with Benjamin Schuster and Joshua Kays in the lab of Professor Justin Hanes in the Biochemical Engineering department of JHU.

The study involved polymeric nanoparticles with a dense polyethylene glycol (PEG) coating to minimize particle adhesion to the gel. The particle core was imbued with photoactivatable (caged) rhodamine, which becomes fluorescent only if the rhodamine is ``uncaged'' through temporary exposure to UV light. This permitted us to selectively photoactivate a region of particles with a brief pulse of UV light, and then observe the spread of the fluorescent particles in the gel over tens of micrometers and tens of minutes, all using confocal microscopy. Complementing these measurements, we were also able to quantify diffusion of the photoactivated particles at high spatiotemporal resolution -- tens of nanometers and tens of milliseconds -- using multiple particle tracking (MPT) on a widefield microscope. MPT has been harnessed recently to measure transport of drug delivery nanoparticles in many biological materials, including brain tumor tissue, mucus, vitreous, and inside cells [cite ADDR review or papers cited therein]. MPT is a powerful technique that permits examination of individual particles and analysis of heterogeneous transport behavior. However, because of the limited depth of field of high-numerical aperture objectives, it is difficult to track particles diffusing in 3D for more than a few seconds and a few micrometers, so another approach is needed to directly observe percolation through biological gels over longer scales. Particle tracking and the photoactivation technique are complementary methods that, together, permit multiscale diffusion measurements. 
[Compare with photoactivation with photobleaching (FRAP) and other techniques, such as diffusion chambers. See notes from Scot Kuo and paper from Bancaud. FRAP papers from Jain and Braeckmans. Discuss select other uses of photoactivation in the literature (probably mostly from cell biology). Cite Politz, Patterson, Lippincot-Schwartz, Danuser(?)]
We first confirmed agreement between measurements from MPT and the photoactivation technique on particles diffusing in water. Then we applied our method to fibrin, a model protein gel system, and found that both MPT and the photoactivation method reveal mobile and immobile populations of particles. Finally, we examined nanoparticle diffusion in sputum collected from cystic fibrosis (CF) patients. Sputum is a major barrier to inhaled CF therapeutics, and our approach enabled us to measure particle diffusion over distances relevant to drug delivery in the lungs.

\section{Experimental Methods}
\subsection{Materials}
Cholalic acid sodium salt (CHA), NVOC2-5-carboxy-Q-rhodamine-NHS ester (caged rhodamine-NHS ester), N-hydroxysulfosuccinimide sodium salt (sulfo-NHS), and N-(3-Dimethylaminopropyl)-N?-ethylcarbodiimide hydrochloride (EDC) were purchased from Sigma-Aldrich (St. Louis, MO). Poly(lactide-co-glycolide(75:25)) amine endcap (PLGA-NH2), Mn 10kDa-15kDa was purchased from Polyscitech (West Lafayette, IN). Poly(lactide-co-glycolide(67:33))-polyethylene glycol (45kDa-5kDa) diblock copolymer (PLGA-PEG) was custom-synthesized by Jinan Daigang Biomaterial Co., Ltd, (Jinan, China). 5 kDa methoxy-PEG-amine was purchased from Creative PEGWorks (Winston Salem, NC). Fluorescent carboxylate-modified polystyrene microspheres (PS-COOH) of diameter 100, 200, and 500 nm were purchased from Molecular Probes (Eugene, Oregon). Human $\alpha$-thrombin (activity 3059 NIH U/mL) and human fibrinogen (plasminogen depleted, activity 100\%) were purchased from Enzyme Research Laboratories (South Bend, IN).
\subsection{Particles}
The particles were designed and synthesized by my collaborators, Benjamin Schuster and Joshua Kays. Their protocol is presented here, for context.
\subsubsection{Labeling of PLGA with caged rhodamine}
Caged rhodamine-NHS ester and PLGA-amine were conjugated through formation of an amide bond. Briefly, 90 mg of PLGA-NH2 was added to 5 mg of caged rhodamine-NHS ester (for a slight molar excess of dye compared to PLGA, 1:1.23) and put under vacuum for 1 h. The mixture was then flushed with nitrogen gas, dissolved in 500 \textmu L of anhydrous dichloromethane (DCM), and reacted for 12 h at room temperature, all under nitrogen gas. Additional DCM was added as needed to facilitate transfer of the product into 10 mL of -20 �C diethyl ether to precipitate the product. The product was washed twice in cold ether by centrifugation. Excess ether was decanted off and the final product was placed in a lyophilizer (FreeZone 4.5 Plus; Labconco) for 12 h. The dried product was stored at  -20 �C  in a shielded container to prevent exposure to incident UV light.
\subsubsection{Particle formulation: PS-PEG} 
PS-PEG particles were prepared as previously described\cite{Nance2012} by coupling PS-COOH with PEG-amine using carbodiimide chemistry1. Generally, 100 ul of stock (2\% solids) PS-COOH particle solution was added to borate buffer (pH 8) with 4 fold molar excess 5k mPEG-amine, EDC, and sulfo-NHS. The particles were reacted for at least 2 h, then washed three times by centrifugation and stored in DI water at 4� C.
\subsubsection{Particle formulation: PLGA-PEG}
PLGA/PLGA-PEG nanoparticles were prepared by using the emulsion method according to the literature\cite{Xu2013}. Briefly, a 40 mg mixture (19:1 by mass) of PLGA-b-PEG5k and PLGA-caged rhodamine was dissolved in 400 \textmu L of DCM, making a 100 mg/mL solution. This solution was injected into an ice-cooled 5 mL of 0.5\% CHA aqueous solution, and sonicated at 30\% amplitude for 2 min using a 130 Watt probe sonicator (Sonics \& Materials, Newtown, CT ). The emulsion was immediately added to 35 mL of 0.5\% CHA solution and stirred at 600 rpm for at least 3 h to allow for complete particle hardening. The final particle suspension was filtered through a 5 \textmu m and then 0.45 \textmu m syringe filter, then the particles were collected and washed three times via centrifugation at 20,000 g for 25 min. 
\subsubsection{Particle Characterization}
The diameter and $\zeta$-potential of the nanoparticles were determined by dynamic light scattering and laser Doppler electrophoresis, respectively, using a Zetasizer Nano ZS90 (Malvern Instruments, Southborough, MA). Transmission Electron Microscopy (TEM) images of dried particles were taken on standard 400 mesh copper TEM grids (TedPella, Redding, CA) with a Hitachi H7600 Electron Microscope. Particles were 160 $\pm$ 10 nm in diameter with a PDI of 0.11 $\pm$ 0.02 and a $\zeta$-potential of -3 $\pm$ 4 mV, where error bars report variation between batches.

\subsection{Sample Preparation}
\subsubsection{Fibrin gel}
Human fibrin gel was made from defrosted aliquots of thrombin and fibrinogen. Appropriate particle concentrations (between 0.005-0.00004\% by mass) for tracking PS-PEG or PLGA-PEG particles were made in 2 U/mL thrombin in PBS, with a total volume of 180 \textmu L. 20 \textmu L of 40 mg/mL fibrinogen was pipetted to the solution, yielding a 4 mg/mL concentration of fibrinogen\cite{Spero2011}. Vortex was immediately applied at high speed for ~2 s to homogenize the solution. 30 \textmu L of the solution was immediately pipetted into a 30 \textmu L well on a glass slide, covered with a glass coverslip, and sealed with a cyanoacrylate glue. The slide was then incubated at 37 �C for 20 min to ensure gelation. In all cases, slides were wrapped in aluminum foil to prevent premature exposure to UV light. 
\subsubsection{CF Sputum}
CFS was collected from adult patients at the Johns Hopkins Cystic Fibrosis Center in accordance with Institutional Review Board-approved protocols. Samples were stored at 4$^\circ$C immediately after collection, and were analyzed the next day. CFS slides were made according to a similar procedure as the above for fibrin gels: 30 \textmu L aliquots of CFS were withdrawn using a Wiretrol (Drummond Scientific Company, Broomall, PA) and injected into 30 \textmu L wells on glass slides. 1 \textmu L of appropriate particle suspensions for tracking or for photoactivation was added to the aliquot and mixed thoroughly. The slide was then sealed and allowed to incubate at room temperature for 1-2 hours (to prevent convection effects).  

\subsection{Measurements}           
\subsubsection{Multiple Particle Tracking (MPT)}
Particle motion in the CF sputum and fibrin samples was observed at room temperature using an inverted epifluorescence microscope (Axio Observer; Carl Zeiss, Thornwood, NY) with a 100X/1.46 NA oil-immersion objective. Movies were collected with 19-ms exposure at 20 fps for 500 frames using an EMCCD camera (Evolve 512; Photometrics, Tucson, AZ). Particle motion was tracked using the particle-tracking software described in Chapter \ref{chap:trackpy}.

\subsubsection{Confocal imaging and photoactivation}
Confocal imaging was performed on a Zeiss LSM510 laser scanning confocal microscope (Carl Zeiss, Thornwood, NY) in multi-tracking mode using either the 63x (Plan Apochromat 1.4 NA), 40x (PlanNeofluar 1.3 NA), or 20x (Plan Apochromat .75 NA) oil-immersion objective. Particles in the selected regions were activated by 10 to 25 iterations of the 405-nm laser at 100\% power, while particles were excited to fluorescence by the 543-nm laser at 10--30\%  power.   All videos were $512\times512$ pixels, and the activated region was $40\times512$ pixels. (For experiments performed using 20X, 40X and 63X objectives, this corresponds to $35\times449$, $18\times225$, or $11\times143$ pixels.)
Slide preparation for confocal experiments was identical to the above procedure for multiple particle tracking in CFS, with the exception of particle concentrations: PLGA-PEG particles were added to either water, CFS, or fibrin with final concentrations of 0.075\% to 0.03\% by mass. 


\section{Results}

\subsection{Water}

Here we obtain the particles' diffusivity $D$ from the evolution of the observed intensity profile $I(x, t)$. Under carefully tuned experimental conditions, intensity is directly proportional to the concentration of activated particles $c(x, t)$. Of course this approximation is not universally applicable: particles overlap and scatter incident light and the emmitted light from the particles around them, so at high concentrations the observed intensity must ultimately saturate. Also, the sensor imposes a sharp limit of the range of observable intensities. A carefully designed experiment, then, must match the particle concentration $c$ and fluorescence (controlled in part by the intensity of incident light) to the sensitivity of camera. In our experiments, departures from the linear approximation $I \propto c$ were measurable but small.

In a viscous liquid such as water---characterized in Figure 2 to illustrate the technique---the concentration profile evolves according to the diffusion equation:

\begin{equation}
 \frac{\partial c}{\partial t} = D\frac{\partial^2 c}{\partial x^2}
\end{equation}

\noindent An idealized, infinitely thin activation region at $x'$ and time $t'$, described by $c(x) = \delta(x-x')$, would spread as a Gaussian that spreads during some time interval $t-t'$ as

\begin{equation}
 c(x, t) = \frac{1}{\sqrt{4\pi D (t-t')}}\exp\left[-(x-x')^2/4D(t - t')\right]
\end{equation}

\noindent This Gaussian is the Green function $G(x - x'; t - t')$ for the diffusion equation. Therefore, for any distribution of activated particles $c(x', t')$ at time $t'$, the distribution at some later time $t$ will be

\begin{equation}
 \label{eqn:convolution}
 c(x, t) = \int_{-\infty}^\infty G(x - x'; t - t')\, c(x', t')\,dx'
\end{equation}

\noindent and therefore, if $I \propto c$,

\begin{equation}
 \label{eqn:convolution}
 I(x, t) = \int_{-\infty}^\infty G(x - x'; t - t')\, I(x', t')\,dx'.
\end{equation}

\noindent That is, the intensity profile at $t'$ can be mapped onto the intensity profile at $t$ through convolution with a Gaussian with width $\sigma=4D(t-t')$. Breakdown of this mapping would indicated non-diffusive motion, which is not expected for water but could occur in other materials.

During an experiment, hundreds of intensity profiles were captured at a regular interval. Every profile was mapped onto each future profile through convolution with a Gaussian, as in Eq. (\ref{eqn:convolution}). The width $\sigma$ of the Gaussian generating the most accurate mapping was determined using a nonlinear least-squares fit. Each mapping constituted a separate---though not strictly statistically independent---measurement of diffusivity $D$. Mappings corresponding to the same time interval $\Delta t = t - t'$ were averaged to produce the data points $\sigma^2(\Delta t)$ shown in Figure 2. Finally, an estimate of $D$ was obtained through a linear regression to $\sigma^2(\Delta t)$; its value, 2.42 $\pm$ ??? \textmu m$^2$/s, is within 5\% of the expected value for the diffusivity of 160-\textmu m particles in water.

\subsection{Fibrin}
\subsection{Cystic Fibrosis Sputum}

\section{Discussion \& Conclusion}

We have developed a strategy to measure nanoparticle diffusion in biological gels over multiple scales, ranging in time from tens of milliseconds to tens of minutes, and ranging in length from less than one micrometer to hundreds of micrometers.
In the model systems tested, results from particle tracking and from the photoactivation technique agree well [or "are compatible"... we'll see!]. In these systems, particle tracking appears capable of [maybe: qualitatively, if not quantitatively,] predicting whether particles can also diffuse over the longer distances relevant to many NP drug delivery applications.
The two techniques offer complimentary strengths. Particle tracking permits analysis of individual particles, and at high spatial and temporal resolutions, but time and length of observation are limited for traditional microscopy setups. The photoactivation technique can monitor diffusion over longer times and distances, which are relevant in many drug delivery (and other?) applications, although spatial and temporal resolutions are not as good. (Compare to FRAP again?)

[Compare sampling bias in the 2 methods.]

We directly observed PEG-coated PLGA nanoparticles diffusing over physiological distances in CFS. This extends prior, related work from the Hanes lab. <Say something here about the Kirch paper.> In the context of lung drug delivery, the diffusive particles may avoid mucociliary clearance and exhibit improved retention and distribution in the lungs.

We observed spatial heterogeneity in particle transport in CFS. This is not evident from particle tracking. <Is this heterogeneity an artifact of sample prep, or does it truly mimic the in vivo situation? I believe an argument can be made for the latter. If so, what are the biomedical implications?>

The bimodal distribution of particle transport in fibrin - which was observed by other investigators at short times by particle tracking - also persists over long times. <What does this say about the mechanism of trapping?>


The technique is generalizable, but of course the results are not generalizable, i.e. method can be used in other gels/fluid/tissue, but the results and correspondence between MPT and photoactivation will of course be case-by-case.
Multiphoton microscopy might be an even cleaner way to do the photoactivation experiment, would only activate in-focus particles, less scattering. Other microscopy techniques that could do this even better? Watch spread in 3D?
Other geometries of activation region?

Future work: Study intracellular NP transport using photoactivable NP. Biologists have done work with photoactivable proteins and small molecules in cells.