% !TEX root = root.tex
\chapter{The Role of Protein Unfolding on Mechanical Evolution of a Protein Layer During its Formation}
\chaptermark{Protein Unfolding}

\section{Introduction}

Previous work on the mechanical properties of protein layers has characterized well-studied proteins like BSA, lysozyme, and ?-lactoglobulin, surveying the evolution of surface pressure and interfacial viscosity during layer formation. While a variety of responses and layer evolutions are observed, it is difficult to conclude why differences are seen, how precisely the differences between these proteins� structure contribute to differences in the formation and the rheology of interfacial layers.

In this paper, we study the layer formation of two proteins that are nearly identical in composition but very different in structure. Specifically, we study SNase and engineered variant of SNase with only one residue changed. That one change catastrophically destabilizes the protein�s folded structure.


\section{Experimental Methods}

\subsection{Protein Fabrication}

Order a plasmid, a ring of DNA. Order a primer, a synthetic fragment of DNA chemically synthesized by joining base pairs, which largely compliments the plasmid but contains a snippet of noncomplemtary DNA. 

Put the primers and plasmids into a soup of loose ribonucleotides and DNA polymerase, which marches along the chain and builds a DNA polymer along complemtary strands. So, the original primers are extended to full strands, and then full copies are made as the process proceeds. This is PCR. It is a decades-old technique.

You subject the bacteria to thermal stress ("heat shock" up to about 40-50 C), and they imbibe the plasmids from the environment, looking for communally shared survival methods. Then the bacteria make the desired protein from the plasmids. Incidentally, synthesizing proteins without help from bacteria or cells is possible but challenging and expensive.

\subsection{Sample Preparation}

\subsection{Active Microrheology}

\section{Results}
\subsection{Wild-Type Layer Evolution}
\subsection{Disordered Layer Evolution}
\section{Discussion}
\section{Conclusion}